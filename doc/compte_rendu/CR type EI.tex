% Document : compte rendu des DM
% Auteur : Riccardo VENTURINI
% Année académique : 2025-2026

\documentclass[a4paper,10pt]{article}

% passe en mode large sur la page A4
\usepackage{a4wide} 

% document francisé
\usepackage[francais]{babel} 

% permet la frappe de caracteres accentues (sur macOS)
\usepackage[utf8x]{inputenc} 
\usepackage{longtable}
\usepackage{amsmath}

\usepackage{listings}
\usepackage{color}

\usepackage{amssymb}
\usepackage{graphicx}

\definecolor{dkgreen}{rgb}{0,0.6,0}
\definecolor{gray}{rgb}{0.5,0.5,0.5}
\definecolor{mauve}{rgb}{0.58,0,0.82}

\lstset{frame=tb,
  language=Julia,
  aboveskip=3mm,
  belowskip=3mm,
  showstringspaces=false,
  columns=flexible,
  basicstyle={\small\ttfamily},
  numbers=none,
  numberstyle=\tiny\color{gray},
  keywordstyle=\color{blue},
  commentstyle=\color{dkgreen},
  stringstyle=\color{mauve},
  breaklines=true,
  breakatwhitespace=true,
  tabsize=3
}

% individualisation des parametres de la page
\parskip8pt
\setlength{\topmargin}{-25mm}
\setlength{\textheight}{250mm}

\usepackage[noend]{algpseudocode}
\usepackage{algorithm}
%
% -----------------------------------------------------------------------------------------------------------------------------------------------------
%

\begin{document}

~
\vspace{50mm}
{\large
\begin{center}
  Nantes Université --- UFR Sciences et Techniques\\
  Master informatique\\
  Année académique 2025-2026
  \vspace{30mm}
 
  { \LARGE
 
     Dossier Exercice d'implémentation\\
     \vspace{5mm}
 
     {\huge \textbf{Métaheuristiques}}
     \vspace{5mm}
 
     Riccardo \textsc{VENTURINI}%$^1$ --  Prénom \textsc{Nom}$^2$
     \vspace{50mm}
  
     \today
  }  
\end{center}
}

\vfill
\break

% -----------------------------------------------------------------------------------------------------------------------------------------------------

% =====================================================================================
% Document : rendu du DM1
% Auteur : Xavier Gandibleux
% Année académique : 2020-2021

\section*{Livrable de l'exercice d'implémentation  1 : \\ Heuristiques de construction et d'amélioration gloutonnes}

%
% -----------------------------------------------------------------------------------------------------------------------------------------------------
%

\vspace{5mm}
\noindent
\fbox{
  \begin{minipage}{0.97 \textwidth}
    \begin{center}
      \vspace{1mm}
      \Large{Formulation du SPP}
      \vspace{1mm}
    \end{center}
  \end{minipage}
}
\vspace{2mm}

\noindent
Soit $U = \{1, 2, \ldots, n\}$ l'ensemble universel et une collection d'ensembles $\mathcal{S} = \{S_1, S_2, \ldots, S_m\}$, où $S_j \subseteq U$, chacun ayant un poids associé $c_j \in {R}^{+}$.

L'objectif est de sélectionner un sous-ensemble d'ensembles de $\mathcal{S}$ qui sont mutuellement disjoints et dont la somme des poids est maximale.

\subsection*{Variables de Décision}
$x_j \in \{0, 1\}, \quad \forall j \in \{1, 2, \ldots, m\}$
$$
x_j =
\begin{cases}
1 & \text{si l'ensemble } S_j \text{ est sélectionné} \\
0 & \text{sinon}
\end{cases}
$$

\subsection*{Formulation du Problème}
\begin{align*}
\text{\textbf{Maximiser}} \quad & \sum_{j=1}^{m} c_j \cdot x_j \\
\text{\textbf{Sous les contraintes :}} \\
& \sum_{j: i \in S_j} x_j \leq 1 \quad & \forall i \in U \tag{Contrainte de Disjonction} \\
& x_j \in \{0, 1\} \quad & \forall j \in \{1, 2, \ldots, m\} \tag{Variables Binaires}
\end{align*}

\subsection*{Cas practique : Le Problème de Planification Ferroviaire (RPP)}
\noindent
Le RPP se concentre sur l'optimisation de la construction ou de la reconstruction d'infrastructures ferroviaires. Il s'agit spécifiquement de planifier l'utilisation et la capacité des composants critiques d'un système ferroviaire.

\noindent \textbf{Étant donné :}
\begin{itemize}
    \item Un ensemble fini d'éléments $I = \{1, \ldots, n\}$
    \item $\{T_j\}$, $j \in J = \{1, \ldots, m\}$, une collection de $m$ sous-ensembles de $I$
\end{itemize}
\noindent
Une sélection (ou recouvrement) est un sous-ensemble $P \subseteq I$ tel que $|T_j \cap P| \leq 1, \forall j \in J$, ce qui mène à la formulation :

\begin{equation*}
\label{eq:SPP_Strict}
\begin{aligned}
\textbf{Maximiser} \quad & z(\mathbf{x}) = \sum_{i \in I} c_i x_i \\
\textbf{Sous les contraintes :} \quad & \sum_{i \in I} t_{i,j} x_i \leq 1 \quad & \forall j \in J \\
& x_i \in \{0, 1\} \quad & \forall i \in I \\
& t_{i,j} \in \{0, 1\} \quad & \forall i \in I, \forall j \in J
\end{aligned}
\end{equation*}

\begin{itemize}
    \item Fortement NP-Difficile (Garey and Johnson 1979)
    \item Problème du Recouvrement de Nœuds : $\sum_{i \in I} t_{i,j} = 2, \forall j \in J$
\end{itemize}

%
% -----------------------------------------------------------------------------------------------------------------------------------------------------
%

\vspace{5mm}
\noindent
\fbox{
  \begin{minipage}{0.97 \textwidth}
    \begin{center}
      \vspace{1mm}
        \Large{Modélisation JuMP (ou GMP) du SPP}
      \vspace{1mm}
    \end{center}
  \end{minipage}
}
\vspace{2mm}

\noindent
La librairie \textbf{JuMP} est un langage de modélisation mathématique open-source conçu pour le langage \textsc{Julia}. Elle permet aux utilisateurs de formuler des problèmes d'optimisation complexes.
Dans cet exemple, nous avons utilisé l'optimiseur GLPK.

\begin{lstlisting}
C, A = loadSPP(fullpath)

    solverSelected = GLPK.Optimizer
    spp = setSPP(C, A)

    set_optimizer(spp, solverSelected)
     t_start = time()
    optimize!(spp)
    time_jump = time() - t_start
\end{lstlisting}

\noindent
Vous trouverez ci-dessous les résultats de l'expérience.

Result GLPK
\begin{longtable}[c]{| c | c | c |}
\hline
File & GLPK (JuMP) & Time (s) \\
\hline
pb\_1000rnd0100.dat & 67 & 212.380985 \\
pb\_1000rnd0800.dat & NA & +10 min.\\
pb\_100rnd0500.dat & 639 & 0.000267 \\
pb\_100rnd1200.dat & 23 & 0.231301 \\
pb\_2000rnd0400.dat & NA & +10 min.\\    
pb\_2000rnd0500.dat & NA & +10 min.\\
pb\_200rnd0300.dat & NA & +10 min.\\
pb\_200rnd1800.dat & 19 & 174.513032 \\
pb\_500rnd0700.dat & NA & +10 min.\\
pb\_500rnd1700.dat & NA & +10 min.\\
\hline
\end{longtable}

%
% -----------------------------------------------------------------------------------------------------------------------------------------------------
%

\vspace{5mm}
\noindent
\fbox{
  \begin{minipage}{0.97 \textwidth}
    \begin{center}
      \vspace{1mm}
        \Large{Instances numériques de SPP}
      \vspace{1mm}
    \end{center}
  \end{minipage}
}
\vspace{2mm}

\noindent
Dans le (Tableau \ref{tab:resultatsei1}) les 10 instances sélectionnées.

  \begin{table}[h!]
        \centering
        \begin{tabular}{|c|l|l|l|l|l|}\hline
 Instance& n° Variables& n° contrantes& Densité (\%)&Max-Uns &Meillere valer connue\\\hline\hline
            pb\_1000rnd0100.dat & 1000& 5000& 2,60& 50&67*\\\hline
            pb\_1000rnd0800.dat & 1000& 1000& 0,60& 10&175*\\\hline
            pb\_100rnd0500.dat & 100& 500& 2& 2&639*\\\hline
            pb\_100rnd1200.dat & 100& 300& 2,97& 4&23*\\\hline
            pb\_2000rnd0400.dat & 2000& 10000& 0,55& 20&32\\ \hline            
            pb\_2000rnd0500.dat & 2000& 2000& 2,55& 100&140\\\hline
            pb\_200rnd0300.dat & 200& 1000& 1& 2&731*\\\hline
            pb\_200rnd1800.dat & 200& 200& 1,50& 4&83*\\\hline
            pb\_500rnd0700.dat & 500& 500& 1,20& 10&1141*\\\hline
            pb\_500rnd1700.dat & 500& 1500& 2,17& 20&192*\\\hline
        \end{tabular}
        \caption{instances sélectionnées}
        \label{tab:resultatsei1}
    \end{table}

%
% -----------------------------------------------------------------------------------------------------------------------------------------------------
%

\vspace{5mm}
\noindent
\fbox{
  \begin{minipage}{0.97 \textwidth}
    \begin{center}
      \vspace{1mm}
        \Large{Heuristique de construction appliquée au SPP}
      \vspace{1mm}
    \end{center}
  \end{minipage}
}
\vspace{2mm}

\noindent
L'algorithme glouton a été utilisé pour la construction heuristique.
L’algorithme glouton sélectionne la première colonne admissible, l’ajoute à la solution, puis avance en ne choisissant que des colonnes qui restent compatibles avec celles déjà sélectionnées.
Il construit une solution initiale de manière séquentielle, en ajoutant progressivement des éléments non conflictuels.

\begin{algorithm}[h!]
    \caption{La construction gloutonne}
    \label{alg:greedy_construction}
    \begin{algorithmic}[1] % [1] pour la numérotation des lignes
        \State $S \leftarrow \emptyset$ \Comment{Initialise la solution comme un ensemble vide}
        \State Initialiser l'ensemble des candidats $C$, et évaluer l'utilité $u(e), \forall e \in C$
        \While {($C \neq \emptyset$)}
            \State Sélectionner le meilleur élément courant $e$ de $C$ :
            \State \quad $e \leftarrow \underset{e \in C}{\operatorname{next}} \, u(e)$ \Comment{Choisir l'élément suivante}
            \State Incorporer $e$ dans la solution :
            \State \quad $S \leftarrow S \cup \{e\}$
            \State Mettre à jour l'ensemble des candidats $C$ et réévaluer l'utilité $u(e), \forall e \in C$
            \State \quad $C \leftarrow C \setminus \text{conflict}(\{e\})$ \Comment{Retirer $e$ et tous les éléments en conflit avec $e$}
        \EndWhile
        \State \textbf{retourner} $S$
    \end{algorithmic}
\end{algorithm}

\begin{itemize}
    \item \textit{S}: ensemble des solutions
    \item \textit{C} : solutions candidates
    \item \textit{e} : variable
\end{itemize}

\subsection*{Exemple Didactique de la Construction Gloutonne (Sélection Forcée)}

\noindent
Considérons un problème de sélection d'éléments pour maximiser l'utilité totale, soumis à des contraintes de conflit.

\textbf{Données :}
\begin{itemize}
    \item \textbf{Ensemble initial des candidats $C$ :} $C = \{e_1, e_2, e_3, e_4, e_5, e_6, e_7\}$.
    \item \textbf{Utilités $u(e)$ :}
    \begin{center}
    \begin{tabular}{|c|c|c|c|c|c|c|c|}
    \hline
    Élément & $e_1$ & $e_2$ & $e_3$ & $e_4$ & $e_5$ & $e_6$ & $e_7$ \\
    \hline
    Utilité $u(e)$ & 10 & 5 & 8 & 6 & 9 & 13& 11 \\
    \hline
    \end{tabular}
    \end{center}
\end{itemize}
\textbf{Déroulement de l'algorithme :}

\begin{enumerate}
    \item \textbf{Initialisation} : $S = \emptyset$. $C = \{e_1, e_2, e_3, e_4, e_5, e_6, e_7\}$.
    \item \textbf{Itération 1 (Sélection de $e_1$)} :
    \begin{itemize}
        \item \textbf{Sélection} : On choisit $e=e_1$ (Utilité 10).
        \item \textbf{Incorporation} : $S \leftarrow \{e_1\}$. Utilité totale = 10.
        \item \textbf{Mise à jour $C$} : $\text{conflict}(\{e_1\}) = \{e_2, e_3, e_5, e_6, e7\}$.
        \item $C \leftarrow C \setminus \{e_2, e_3, e_5, e_6, e7\}$.
        \item Nouveau $C = \{e_4\}$.
    \end{itemize}
    
    \item \textbf{Itération 2 (Gloutonne)} :
    \begin{itemize}
        \item \textbf{Sélection} : Dans $C=\{e_4(6)\}$, $\max_{e \in C} u(e) = u(e_4) = 6$. L'élément choisi est $e=e_4$.
        \item \textbf{Incorporation} : $S \leftarrow \{e_1, e_6\}$. Utilité totale = $10 + 6 = 16$.
        \item \textbf{Mise à jour $C$} : $\text{conflict}(\{e_4\}) = \{e_4\}$.
        \item $C \leftarrow C \setminus \{e_4\}$.
        \item Nouveau $C = \emptyset$.
    \end{itemize}

    \item \textbf{Arrêt} : La boucle \textbf{While} se termine car $C = \emptyset$.
    
    \item \textbf{Résultat final} : La solution gloutonne retournée est $S = \{e_1, e_4\}$, avec une utilité totale de 16.
\end{enumerate}
%
% -----------------------------------------------------------------------------------------------------------------------------------------------------
%

\vspace{5mm}
\noindent
\fbox{
  \begin{minipage}{0.97 \textwidth}
    \begin{center}
      \vspace{1mm}
        \Large{Heuristique d'amélioration appliquée au SPP}
      \vspace{1mm}
    \end{center}
  \end{minipage}
}
\vspace{2mm}

\noindent


\begin{algorithm}[h!]
    \caption{Procédure de Descente Simple}
    \label{alg:simple_descent}
    \begin{algorithmic}[1] % [1] pour la numérotation des lignes
        \Procedure{DescenteSimple}{solution initiale $x \in X$}
            \Repeat
                \State choisir $x' \in N(x)$ \Comment{Sélectionner un voisin $x'$ de $x$}
                \If {$z(x') < z(x)$} \Comment{Si le voisin $x'$ est meilleur (minimisation)}
                    \State $x \leftarrow x'$ \Comment{Déplacer la solution courante vers ce meilleur voisin}
                \EndIf
            \Until {$f(x') \geq f(x), \forall x' \in N(x)$} \Comment{Jusqu'à ce que $x$ soit un minimum local}
            \State \textbf{Retourner} $x$
        \EndProcedure
    \end{algorithmic}
\end{algorithm}


\begin{itemize}
    \item X : l'espace de toutes les solutions réalisables (l'espace de recherche)
    \item N(x) : le voisinage de la solution courante $x$
    \item x' : une solution voisine candidate, appartenant au voisinage $N(x)$
    \item z(x) : la fonction objectif de la solution $x$
    \item f(x) : la condition d'arrêt qui signifie que la procédure s'arrête lorsque la solution courante $x$ est un optimum local, c'est-à-dire qu'aucune solution voisine $x'$ n'est meilleure que $x$
\end{itemize}

\subsection*{Exemple Didactique de la Descente Simple}

\textbf{Déroulement de l'algorithme (avec $x_{initial} = \{1, 4\}$) :}

\begin{enumerate}
    \item \textbf{Étape 1 : Initialisation.}
    \begin{itemize}
        \item Solution courante : $x = \{1, 4\}$.
        \item Coût courant : $z(x) = c_1 + c_4 = 10 + 6 = 16$.
    \end{itemize}

    \item \textbf{Étape 2 : Exploration (Boucle Repeat 1).}
    \begin{itemize}
        \item Voisinage $N(x)$ (Ajout ou Retrait d'un élément) :
        \begin{itemize}
            \item Retrait : $x'_a = \{1\}$, $z(x'_a)=10$. NON ($10 < 16$).
            \item Retrait : $x'_b = \{4\}$, $z(x'_b)=6$. NON ($6 < 16$).
            \item Ajout : $x'_c = \{1, 4, 5\}$, $z(x'_c)=10+6+9=25$. OUI ($25 > 16$).
        \end{itemize}
        \item Voisin choisi (le meilleur trouvé ou le premier trouvé meilleur) : $x' = \{1, 4, 5\}$.
        \item Mise à jour : $x \leftarrow \{1, 4, 5\}$.
    \end{itemize}

    \item \textbf{Étape 3 : Exploration (Boucle Repeat 2).}
    \begin{itemize}
        \item Solution courante : $x = \{1, 4, 5\}$, $z(x) = 25$.
        \item Voisinage $N(x)$ (Ajout ou Retrait) :
        \begin{itemize}
            \item Retrait : $x'_d = \{4, 5\}$, $z(x'_d)=6+9=15$. NON.
            \item Ajout : $x'_e = \{1, 4, 5, 6\}$, $z(x'_e)=10+6+9+13=38$. OUI ($38 > 25$).
        \end{itemize}
        \item Voisin choisi : $x' = \{1, 4, 5, 6\}$.
        \item Mise à jour : $x \leftarrow \{1, 4, 5, 6\}$.
    \end{itemize}

    \item \textbf{Étape 4 : Exploration (Boucle Repeat 3).}
    \begin{itemize}
        \item Solution courante : $x = \{1, 4, 5, 6\}$, $z(x) = 38$.
        \item Voisinage $N(x)$ (Ajout ou Retrait) :
        \begin{itemize}
            \item Retrait : $x'_f = \{1, 5, 6\}$, $z(x'_f)=10+9+13=32$. NON.
            \item Ajout : $x'_g = \{1, 4, 5, 6, 7\}$, $z(x'_g)=10+6+9+13+11=49$. OUI ($49 > 38$).
        \end{itemize}
        \item Voisin choisi : $x' = \{1, 4, 5, 6, 7\}$.
        \item Mise à jour : $x \leftarrow \{1, 4, 5, 6, 7\}$.
    \end{itemize}
    
    \item \textbf{Étape 5 : Exploration (Boucle Repeat 4).}
    \begin{itemize}
        \item Solution courante : $x = \{1, 4, 5, 6, 7\}$, $z(x) = 49$.
        \item Voisinage $N(x)$ (Retrait d'un élément) :
        \begin{itemize}
            \item Retrait de $1$ : $z=39$. NON.
            \item Retrait de $4$ : $z=43$. NON.
            \item Retrait de $7$ : $z=38$. NON.
        \end{itemize}
        \item \textbf{Condition d'arrêt atteinte :} Aucune solution voisine n'a un coût supérieur à $z(x)=49$.
    \end{itemize}

    \item \textbf{Résultat :} L'algorithme s'arrête en $x=\{1, 4, 5, 6, 7\}$ avec un coût $z(x)=49$. Cette solution est un \textbf{maximum local}.
\end{enumerate}

%
% -----------------------------------------------------------------------------------------------------------------------------------------------------
%

\vspace{5mm}
\noindent
\fbox{
  \begin{minipage}{0.97 \textwidth}
    \begin{center}
      \vspace{1mm}
        \Large{Expérimentation numérique}
      \vspace{1mm}
    \end{center}
  \end{minipage}
}
\vspace{2mm}

\noindent
Machine sur laquelle les résultats ont été enregistrés : 

MacMini M4 (ARM) - CPU 10 cœurs

Result E1
\begin{longtable}[c]{| c | c | c | c | c |}
\hline
File & Heuristic & Time (s) & Local Search & Time (s) \\
\hline
pb\_1000rnd0100.dat & 22 & 0.156478 & 40 & 5.488210 \\
pb\_1000rnd0800.dat & 108 & 0.001138 & 108 & 25.506977 \\
pb\_100rnd0500.dat & 533 & 0.000038 & 620 & 0.025510 \\
pb\_100rnd1200.dat & 17 & 0.000039 & 17 & 0.012132 \\
pb\_2000rnd0400.dat & 20 & 0.051748 & 20 & 222.783113 \\
pb\_2000rnd0500.dat & 36 & 0.005408 & 91 & 30.030356 \\
pb\_200rnd0300.dat & 424 & 0.000159 & 662 & 4.381937 \\
pb\_200rnd1800.dat & 12 & 0.000074 & 12 & 0.073140 \\
pb\_500rnd0700.dat & 667 & 0.000266 & 975 & 38.437973 \\
pb\_500rnd1700.dat & 98 & 0.000663 & 137 & 3.493589 \\
\hline
\end{longtable}



%
% -----------------------------------------------------------------------------------------------------------------------------------------------------
%

\vspace{5mm}
\noindent
\fbox{
  \begin{minipage}{0.97 \textwidth}
    \begin{center}
      \vspace{1mm}
        \Large{Discussion}
      \vspace{1mm}
    \end{center}
  \end{minipage}
}
\vspace{2mm}

\noindent
Dans ces exemples, l'utilisation de GLPK n'est pas justifiée car, dans la plupart des cas, sa résolution prend plus de 10 minutes, et obtenir de bonnes solutions n'est pas seulement une question de résultats, mais aussi de temps.
En revanche, mon heuristique trouve une solution beaucoup plus rapidement, mais nous sommes encore un peu loin de l'optimalité. Avec la recherche locale par échange 1-1, nous nous approchons dans un délai acceptable. J'ai choisi cette implémentation pour sa simplicité et, surtout, pour sa priorité donnée au gain de temps.
Ainsi, comparée à GLPK, elle est préférable car au moins, nous avons toujours une solution.

\noindent
Je suppose que l'utilisation de métaheuristiques pourrait être prometteuse , il existe certainement des solutions plus éloignées, difficiles à explorer avec les techniques de voisinage classiques.

\noindent
Dans mon exemple, je n'utilise pas l'algorithme de Glouton traditionnel, qui ne sélectionne pas la valeur maximale dans l'espace des candidats, mais la suivante admissible. La solution initiale importe peu, c'est le chemin d'exploration qui compte pour trouver la solution optimale.

\noindent
Dans ce cas, nous n'avons pas d'aléatoire, une piste d'amélioration pourrait consister à réduire le déterminisme, comme c'était le cas dans GRASP.
 

% -----------------------------------------------------------------------------------------------------------------------------------------------------

% =====================================================================================
% Document : rendu du IE2
% Auteur : Xavier Gandibleux
% Année académique : 2024-2025

\section*{Livrable de l'exercice d'implémentation 2 : \\ Métaheuristique GRASP, ReactiveGRASP et extensions}

%
% -----------------------------------------------------------------------------------------------------------------------------------------------------
%

\vspace{5mm}
\noindent
\fbox{
  \begin{minipage}{0.97 \textwidth}
    \begin{center}
      \vspace{1mm}
        \Large{Présentation succincte de GRASP appliqué sur le SPP}
      \vspace{1mm}
    \end{center}
  \end{minipage}
}
\vspace{2mm}

\noindent
Présenter l'algorithme mis en oeuvre. Illustrer sur un exemple didactique (poursuivre avec l'exemple pris en DM1). Présenter vos choix de mise en oeuvre.

 La méthode GRASP (Greedy Randomized Adaptive Search Procedure) est une métaheuristique que combine les méthodes gloutonnes et aléatoires.
 La construction d'une solution se déroule par étapes et à chacune de celles-ci, l'ensemble des morceaux de solution qu'il est possible d'ajouter est placé dans une liste appelée RCL (Restricted Candidate List). Dans la partie gloutonne cette liste est triée, mais ce n'est pas nécessairement le meilleur morceau qui est ajouté à la solution courante. 
 Pour la partie aleatoire on tire aléatoirement parmi les meilleurs possibilités le morceau à ajouter, ça permet donc de varier la forme des solutions générées mais celles-ci sont quand même de bonne qualité, puisque le choix aléatoire se fait parmi un ensemble de bons candidats. 
 La recherche locale s'applique sur la solution réalisable résultante de la phase de construction afin de voir s'il est encore possible d'améliorer cette solution.

 Je n'analyserai ci-dessous que la partie construction, car pour l'amélioration, on utilise la recherche d'échanges locaux 1-1 vue en EI1.

\begin{algorithm}[h!]
    \caption{La construction gloutonne randomisée}
    \label{alg:greedy_randomized_construction}
    \begin{algorithmic}[1]
        \State $S \leftarrow \emptyset$ \Comment{Solution courante}
        \State Initialiser l'ensemble des candidats $C$, et évaluer $u(e), \forall e \in C$
        \While {($C \neq \emptyset$)}
            \State Construire la Liste des Candidats Restreints (RCL) :
            \State \quad $u_{min} \leftarrow \min_{e \in C} u(e)$
            \State \quad $u_{max} \leftarrow \max_{e \in C} u(e)$
            \State \quad $u_{Limit} \leftarrow u_{min} + \alpha \times (u_{max} - u_{min})$ \Comment{$\alpha \in [0, 1]$ est le paramètre de gloutonnerie}
            \State \quad $RCL \leftarrow \{e \in C \mid u(e) \geq u_{Limit}\}$ \Comment{RCL contient les éléments ``suffisamment bons''}
            \State Sélectionner un élément $e$ du RCL au hasard :
            \State \quad $e \leftarrow \text{RandomSelect}(RCL)$
            \State Incorporer $e$ dans la solution :
            \State \quad $S \leftarrow S \cup \{e\}$
            \State Mettre à jour l'ensemble des candidats $C$ :
            \State \quad $C \leftarrow C \setminus \text{conflict}(\{e\})$
        \EndWhile
        \State \textbf{retourner} la solution construite $S$
    \end{algorithmic}
\end{algorithm}

\textbf{Note sur le paramètre $\alpha$} :
\begin{itemize}
    \item Si $\alpha = 0$, Il n'y a pas d'aleatoire.
    \item Si $\alpha = 1$, la sélection est totalement randomisée.
\end{itemize}

\subsection*{Exemple Didactique du GRASP}

\subsubsection*{Données de l'Instance}

\textbf{Éléments $I$ et Utilités $u(e)$ :}
$$I = \{e_1, \ldots, e_9\}$$
\begin{center}
% Correction: 10 colonnes (1 pour le titre + 9 pour e1 à e9)
\begin{tabular}{|c||c|c|c|c|c|c|c|c|c|}\hline
Élément & $e_1$ & $e_2$ & $e_3$ & $e_4$ & $e_5$ & $e_6$ & $e_7$ & $e_8$ & $e_9$ \\\hline
Utilité $u(e)$ & 10 & 5 & 8 & 6 & 9 & \textbf{13} & 11 & 4 & 6 \\\hline
\end{tabular}
\end{center}
\textbf{Contraintes de Conflit (Sous-ensembles $\mathcal{T}$)} :
\begin{itemize}
    \item $T_1 = \{e_1, e_2, e_3, e_5, e_7, e_8\}$
    \item $T_2 = \{e_2, e_3, e_8\}$
    \item $T_3 = \{e_2, e_5, e_6, e_8, e_9\}$
    \item $T_4 = \{e_4\}$
    \item $T_5 = \{e_1, e_3, e_5, e_6, e_9\}$
    \item $T_6 = \{e_2, e_3, e_7, e_9\}$
    \item $T_7 = \{e_1, e_4, e_5, e_8, e_9\}$
\end{itemize}

\subsubsection*{1. Phase de Construction Gloutonne Randomisée ($\boldsymbol{\alpha = 0.5}$)}

L'objectif est de construire une solution initiale $S$ en utilisant le critère de la Liste des Candidats Restreints (RCL).
$$ u_{Limit} = u_{min} + \alpha \times (u_{max} - u_{min}) $$

\subsubsection*{Itération 1}
\begin{itemize}
    \item $C = \{e_1, \ldots, e_9\}$. $u_{max} = 13$ ($e_6$), $u_{min} = 4$ ($e_8$).
    \item $u_{Limit} = 4 + 0.5 \times (13 - 4) = 4 + 4.5 = \mathbf{8.5}$.
    \item $\mathbf{RCL} = \{e \in C \mid u(e) \geq 8.5\} = \{e_1(10), e_5(9), e_6(13), e_7(11)\}$.
    \item \textbf{Sélection Aléatoire (Hypothèse)} : On choisit $\mathbf{e_7}$ (Utilité 11).
    \item \textbf{Mise à jour $S$} : $S = \{e_7\}$. Utilité Totale : 11.
    \item \textbf{Candidats retirés} : $e_7$ est dans $T_1$ et $T_6$. Tous les éléments de $T_1$ et $T_6$ (sauf $e_7$) sont retirés, plus $e_7$.
    $$ \text{Retiré} = \{e_7\} \cup (T_1 \setminus \{e_7\}) \cup (T_6 \setminus \{e_7\}) = \{e_7, e_1, e_2, e_3, e_5, e_8, e_9\} $$
    \item \textbf{Nouveau $C$} : $\{e_4(6), e_6(13)\}$.
\end{itemize}

\subsubsection*{Itération 2}
\begin{itemize}
    \item $C = \{e_4(6), e_6(13)\}$. $u_{max} = 13$ ($e_6$), $u_{min} = 6$ ($e_4$).
    \item $u_{Limit} = 6 + 0.5 \times (13 - 6) = 6 + 3.5 = \mathbf{9.5}$.
    \item $\mathbf{RCL} = \{e \in C \mid u(e) \geq 9.5\} = \{e_6(13)\}$.
    \item \textbf{Sélection Aléatoire} : On choisit $\mathbf{e_6}$ (Utilité 13).
    \item \textbf{Mise à jour $S$} : $S = \{e_7, e_6\}$. Utilité Totale : $11 + 13 = 24$.
    \item \textbf{Candidats retirés} : $\{e_6\}$. (Les conflits avec $e_6$ sont déjà retirés dans $C$).
    \item \textbf{Nouveau $C$} : $\{e_4(6)\}$.
\end{itemize}

\subsubsection*{Itération 3}
\begin{itemize}
    \item $C = \{e_4(6)\}$. $u_{max} = 6, u_{min} = 6$. $u_{Limit} = 6$.
    \item $\mathbf{RCL} = \{e_4(6)\}$.
    \item \textbf{Sélection Aléatoire} : On choisit $\mathbf{e_4}$ (Utilité 6).
    \item \textbf{Mise à jour $S$} : $S = \{e_7, e_6, e_4\}$. Utilité Totale : $24 + 6 = 30$.
    \item \textbf{Candidats retirés} : $\{e_4\}$.
    \item \textbf{Nouveau $C$} : $\emptyset$.
\end{itemize}

\textbf{Solution de Construction $S$} : $S = \{e_4, e_6, e_7\}$, avec une utilité de $\mathbf{30}$.

\subsubsection*{2. Phase d'Amélioration Locale (déjà analisé)}
%
% -----------------------------------------------------------------------------------------------------------------------------------------------------
%

\vspace{5mm}
\noindent
\fbox{
  \begin{minipage}{0.97 \textwidth}
    \begin{center}
      \vspace{1mm}
        \Large{Présentation succincte de Path-Relinking appliqué sur le SPP}
      \vspace{1mm}
    \end{center}
  \end{minipage}
}
\vspace{2mm}

\noindent
\begin{algorithm}
\caption{Path Relinking}
\label{alg:path_relinking}
\begin{algorithmic}[h!]
    \Ensure Meilleure Solution $\mathbf{x}_{\text{best}}$ trouvée sur le chemin, et sa valeur $z_{\text{best}}$.

    \State $\mathbf{x}_i \gets \text{copy}(\mathbf{x}_A)$ \Comment{Initialiser la solution courante}
    \State $\mathbf{x}_{\text{best}} \gets \text{copy}(\mathbf{x}_i)$
    \State $z_{\text{best}} \gets \sum(C \cdot \mathbf{x}_i)$

    \State $\text{Diff} \gets \{ i \mid \mathbf{x}_i[i] \neq \mathbf{x}_B[i] \}$ \Comment{Identifier les indices de différence}

    \While{$\text{Diff} \neq \emptyset$}
        \State Sélectionner aléatoirement un indice $i$ dans $\text{Diff}$
        \State $\mathbf{x}_i[i] \gets \mathbf{x}_B[i]$ \Comment{Appliquer le mouvement de la solution guide $\mathbf{x}_B$}
        
        \State $z_{i} \gets \sum(C \cdot \mathbf{x}_i)$
        
        \If{$z_i > z_{\text{best}}$}
            \State $\mathbf{x}_{\text{best}} \gets \text{copy}(\mathbf{x}_i)$
            \State $z_{\text{best}} \gets z_{i}$
        \EndIf
        
        \State \textbf{Optionnel :} Appliquer une Recherche Locale sur $\mathbf{x}_i$
        \State $(\mathbf{x}_{\text{LS}}, z_{\text{LS}}) \gets \text{localSearch\_1\_1}(C, A, \mathbf{x}_i)$
        
        \If{$z_{\text{LS}} > z_{\text{best}}$}
            \State $\mathbf{x}_{\text{best}} \gets \text{copy}(\mathbf{x}_{\text{LS}})$
            \State $z_{\text{best}} \gets z_{\text{LS}}$
        \EndIf

        \State $\text{Diff} \gets \{ i \mid \mathbf{x}_i[i] \neq \mathbf{x}_B[i] \}$ \Comment{Mettre à jour les indices de différence}
    \EndWhile
    
    \State Ajouter $(\mathbf{x}_{\text{best}}, z_{\text{best}})$ à $\text{EliteSet}$ \Comment{Mettre à jour l'ensemble élite}

    \State \Return $\mathbf{x}_{\text{best}}, z_{\text{best}}$
\end{algorithmic}
\end{algorithm}

\subsubsection*{Données et Solutions Élite}

L'instance SPP comporte 9 éléments ($e_1$ à $e_9$) avec les utilités suivantes :

\begin{center}
\begin{tabular}{|c||c|c|c|c|c|c|c|c|c|c|}\hline
Élément & $e_1$ & $e_2$ & $e_3$ & $e_4$ & $e_5$ & $e_6$ & $e_7$ & $e_8$ & $e_9$ & \textbf{Total} \\\hline
Utilité $u(e)$ & 10 & 5 & 8 & 6 & 9 & 13 & 11 & 4 & 6 & \\\hline
\end{tabular}
\end{center}

Nous sélectionnons les deux solutions élites suivantes (représentées par un vecteur binaire où $1$ signifie sélectionné, $0$ non sélectionné) :

\begin{itemize}
    \item \textbf{Solution de Démarrage $\mathbf{x}_A$} : $\{e_4, e_6, e_7\}$.
    $$ \mathbf{x}_A = (0, 0, 0, 1, 0, 1, 1, 0, 0) \quad \implies z_A = 6 + 13 + 11 = \mathbf{30} $$
    \item \textbf{Solution de Guidage $\mathbf{x}_B$} : $\{e_1, e_2, e_8, e_9\}$.
    $$ \mathbf{x}_B = (1, 1, 0, 0, 0, 0, 0, 1, 1) \quad \implies z_B = 10 + 5 + 4 + 6 = \mathbf{25} $$
    \item \textbf{Meilleure Solution trouvée} : $\mathbf{x}_{\text{best}} \leftarrow \mathbf{x}_A$, $z_{\text{best}} \leftarrow 30$.
\end{itemize}

\subsubsection*{Déroulement du Path Relinking}

\subsubsection*{Initialisation}
L'ensemble des indices de différence ($\text{Diff}$) est :
$$\text{Diff} = \{ i \mid \mathbf{x}_A[i] \neq \mathbf{x}_B[i] \} = \{1, 2, 4, 6, 7, 8, 9\}$$
(7 différences à résoudre pour transformer $\mathbf{x}_A$ en $\mathbf{x}_B$).

\subsubsection*{Itération 1 : Résolution de la différence $e_7$}
\begin{itemize}
    \item \textbf{Sélection Aléatoire} : $i=7$ (élément $e_7$).
    \item \textbf{Mouvement} : $e_7$ est dans $\mathbf{x}_A$ (1) mais pas dans $\mathbf{x}_B$ (0). On retire $e_7$.
    \item \textbf{Nouvelle Solution $\mathbf{x}_i$} : $(0, 0, 0, 1, 0, 1, \mathbf{0}, 0, 0)$.
    \item \textbf{Utilité $z_i$} : $30 - 11 = 19$.
    \item \textbf{Mise à Jour} : $19 \ngtr z_{\text{best}}$. Aucune amélioration.
    \item \textbf{Diff} mis à jour : $\text{Diff} = \{1, 2, 4, 6, 8, 9\}$.
\end{itemize}

\subsubsection*{Itération 2 : Résolution de la différence $e_1$}
\begin{itemize}
    \item \textbf{Sélection Aléatoire} : $i=1$ (élément $e_1$).
    \item \textbf{Mouvement} : $e_1$ n'est pas dans $\mathbf{x}_i$ (0) mais est dans $\mathbf{x}_B$ (1). On ajoute $e_1$.
    \item \textbf{Nouvelle Solution $\mathbf{x}_i$} : $(\mathbf{1}, 0, 0, 1, 0, 1, 0, 0, 0)$ (soit $\{e_1, e_4, e_6\}$).
    \item \textbf{Utilité $z_i$} : $19 + 10 = 29$.
    \item \textbf{Mise à Jour} : $29 \ngtr z_{\text{best}}$. Aucune amélioration.
    \item \textbf{Diff} mis à jour : $\text{Diff} = \{2, 4, 6, 8, 9\}$.
\end{itemize}

\subsubsection*{Itération 3 : Résolution de la différence $e_9$}
\begin{itemize}
    \item \textbf{Sélection Aléatoire} : $i=9$ (élément $e_9$).
    \item \textbf{Mouvement} : On ajoute $e_9$.
    \item \textbf{Nouvelle Solution $\mathbf{x}_i$} : $(1, 0, 0, 1, 0, 1, 0, 0, \mathbf{1})$ (soit $\{e_1, e_4, e_6, e_9\}$).
    \item \textbf{Utilité $z_i$} : $29 + 6 = \mathbf{35}$.
    \item \textbf{Mise à Jour} : $35 > z_{\text{best}}=30$.
        $$\mathbf{x}_{\text{best}} \leftarrow \{e_1, e_4, e_6, e_9\}$$
        $$z_{\text{best}} \leftarrow 35$$
    \item \textbf{Diff} mis à jour : $\text{Diff} = \{2, 4, 6, 8\}$.
\end{itemize}

\subsubsection*{Itération 4 : Résolution de la différence $e_6$}
\begin{itemize}
    \item \textbf{Sélection Aléatoire} : $i=6$ (élément $e_6$).
    \item \textbf{Mouvement} : $e_6$ est dans $\mathbf{x}_i$ (1) mais pas dans $\mathbf{x}_B$ (0). On retire $e_6$.
    \item \textbf{Nouvelle Solution $\mathbf{x}_i$} : $(1, 0, 0, 1, 0, \mathbf{0}, 0, 0, 1)$ (soit $\{e_1, e_4, e_9\}$).
    \item \textbf{Utilité $z_i$} : $35 - 13 = 22$.
    \item \textbf{Mise à Jour} : $22 \ngtr z_{\text{best}}$. Aucune amélioration.
    \item \textbf{Diff} mis à jour : $\text{Diff} = \{2, 4, 8\}$.
\end{itemize}

\subsubsection*{Itérations Finales}
Le processus se poursuit jusqu'à ce que $\mathbf{x}_i$ soit identique à $\mathbf{x}_B$.

\begin{itemize}
    \item \textbf{Itération 5} : Résoudre $e_4$ (Retrait). $\mathbf{x}_i \leftarrow \{e_1, e_9\}$. $z_i = 22 - 6 = 16$.
    \item \textbf{Itération 6} : Résoudre $e_2$ (Ajout). $\mathbf{x}_i \leftarrow \{e_1, e_2, e_9\}$. $z_i = 16 + 5 = 21$.
    \item \textbf{Itération 7} : Résoudre $e_8$ (Ajout). $\mathbf{x}_i \leftarrow \{e_1, e_2, e_8, e_9\}$. $z_i = 21 + 4 = 25$.
\end{itemize}

\subsubsection*{Conclusion du Chemin}
La solution finale du chemin est $\mathbf{x}_i = \mathbf{x}_B$ avec une utilité de 25.
%
% -----------------------------------------------------------------------------------------------------------------------------------------------------
%

\vspace{5mm}
\noindent
\fbox{
  \begin{minipage}{0.97 \textwidth}
    \begin{center}
      \vspace{1mm}
        \Large{Expérimentation numérique de GRASP}
      \vspace{1mm}
    \end{center}
  \end{minipage}
}
\vspace{2mm}

\noindent
Paramètres de test pour les graphiques :
 \begin{itemize}
     \item nbInstances       = 3
     \item nbRunGrasp        = 30
     \item nbIterationGrasp  = 100
     \item nbDivisionRun     = 10
     \item alpha = [0.0, 0.25, 0.5, 0.75, 1.0]
 \end{itemize}
 \begin{figure}[h!]
    \centering
    \begin{minipage}[b]{0.48\linewidth}
        \centering
        \includegraphics[width=\linewidth]{imgs/bilan_tous_runs_GRASP.png}
        \caption{Bilan sur tous les run}
        \label{fig:runGRASP}
    \end{minipage}
    \hfill
    \begin{minipage}[b]{0.48\linewidth}
        \centering
        \includegraphics[width=\linewidth]{imgs/bilan_CPUt_tous_runs_GRASP.png}
        \caption{CPU temps d'execution}
        \label{fig:bilanRuns}
    \end{minipage}
\end{figure}
 \begin{figure}[h!]
     \centering
     \includegraphics[width=1\linewidth]{imgs/GRASP.png}
     \caption{Comparison changement d'alpha sur GRASP}
     \label{fig:placeholder}
 \end{figure}

\vspace{40mm}
\noindent
Paramètres pour test sur best value :
\begin{itemize}
    \item alpha = 0,8
    \item itérations = 5
\end{itemize}
\vspace{2cm}
\begin{longtable}{|c|c|c|}
\hline
Fichier & Solution trouvée & Temps (s) \\
\hline
pb\_1000rnd0100.dat & 56 & 17.461596 \\
pb\_1000rnd0800.dat & 126 & 135.841186 \\
pb\_100rnd0500.dat & 627 & 0.033512 \\
pb\_100rnd1200.dat & 18 & 0.055464 \\
pb\_2000rnd0400.dat & 20 & 1094.570787 \\
pb\_2000rnd0500.dat & 119 & 68.136535 \\
pb\_200rnd0300.dat & 682 & 5.037251 \\
pb\_200rnd1800.dat & 14 & 0.377953 \\
pb\_500rnd0700.dat & 987 & 42.822087 \\
pb\_500rnd1700.dat & 164 & 6.648394 \\
\hline
\end{longtable}

Si on change alpha à 0,3 par exemple : 

\begin{longtable}{|c|c|c|}
\hline
Fichier & Solution trouvée & Temps (s) \\
\hline
pb\_1000rnd0100.dat & 40 & 32.844663 \\
pb\_1000rnd0800.dat & 130 & 135.449871 \\
pb\_100rnd0500.dat & 633 & 0.083169 \\
pb\_100rnd1200.dat & 18 & 0.058411 \\
pb\_2000rnd0400.dat & 21 & 1075.133779 \\
pb\_2000rnd0500.dat & 108 & 137.884737 \\
pb\_200rnd0300.dat & 672 & 12.449107 \\
pb\_200rnd1800.dat & 13 & 0.356583 \\
pb\_500rnd0700.dat & 1023 & 159.023518 \\
pb\_500rnd1700.dat & 145 & 9.482767 \\
\hline
\end{longtable}

Si l'on augmente le déterminisme, GRASP n'explorera pas grand-chose en dehors des résultats trouvés, au contraire, avec trop d'aléatoire, il explorera trop loin. Il est nécessaire de doser correctement alpha en fonction du type de problème.

Ex alpha sur pb\_200rnd0300 :
\begin{itemize}
    \item alpha 1 : 639 en 2 secondes
    \item alpha 0 : 678 en 24 secondes
\end{itemize}

En conclusion, comme nous pouvons le voir sur le graphique, plus nous augmentons la valeur alpha, moins il y aura d'exploration. C'est pourquoi avoir un alpha faible prend plus de temps à s'exécuter.

%
% -----------------------------------------------------------------------------------------------------------------------------------------------------
%

\vspace{5mm}
\noindent
\fbox{
  \begin{minipage}{0.97 \textwidth}
    \begin{center}
      \vspace{1mm}
        \Large{Expérimentation numérique de Path-Relinking}
      \vspace{1mm}
    \end{center}
  \end{minipage}
}
\vspace{2mm}

Paramètres de test pour les graphiques :
\begin{itemize}
    \item nbInstances = 2
    \item nbRunGrasp = 3
    \item nbIterationGrasp = 100
    \item nbDivisionRun = 10
    \item alpha = [0.0, 0.25, 0.5, 0.75, 1.0]
\end{itemize}
\begin{figure}[h!]
    \centering
    \begin{minipage}[b]{0.48\linewidth}
        \centering
        \includegraphics[width=\linewidth]{imgs/bilan_tous_runs_PR.png}
        \caption{Bilan sur tous les run GRASP+PR}
        \label{fig:runGRASP_PR}
    \end{minipage}
    \hfill
    \begin{minipage}[b]{0.48\linewidth}
        \centering
        \includegraphics[width=\linewidth]{imgs/bilan_CPUt_tous_runs_PR.png}
        \caption{CPU temps d'execution GRASP+PR}
        \label{fig:bilanGRASP_PR}
    \end{minipage}
\end{figure}
\begin{figure}[h!]
    \centering
    \includegraphics[width=1\linewidth]{imgs/GRASP_PR.png}
    \caption{Comparison changement d'alpha sur GRASP+PR}
    \label{fig:placeholder}
\end{figure}

\vspace{40cm}

\noindent
Paramètres pour les tests :
\begin{itemize}
    \item alpha = 0,8
    \item itérations = 5
\end{itemize}

\begin{longtable}{|c|c|c|c|c|}
\hline
Fichier & zmin & zmoy & zmax & Temps (s) \\
\hline
pb\_1000rnd0100.dat & 37 & 52.0 & 60 & 55.17 \\
pb\_100rnd0500.dat & 620 & 643.0 & 648 & 0.19 \\
pb\_100rnd1200.dat & 13 & 19.2 & 22 & 1.41 \\
pb\_200rnd0300.dat & 643 & 735.4 & 810 & 82.71 \\
pb\_200rnd1800.dat & 11 & 16.2 & 18 & 8.87 \\
pb\_500rnd1700.dat & 98 & 166.2 & 211 & 53.92 \\
\hline
\end{longtable}

\noindent
Le Path Relinking améliore considérablement les solutions trouvées par GRASP, mais prend beaucoup plus de temps, c'est pourquoi je n'ai pas les résultats pour tous les cas.


%
% -----------------------------------------------------------------------------------------------------------------------------------------------------
%

\vspace{5mm}
\noindent
\fbox{
  \begin{minipage}{0.97 \textwidth}
    \begin{center}
      \vspace{1mm}
        \Large{Discussion}
      \vspace{1mm}
    \end{center}
  \end{minipage}
}
\vspace{2mm}

\noindent

Le GRASP seul approche déjà une solution optimale dans un temps relativement rapide, en veillant à bien paramétrer le nombre d'itérations et $\alpha$. Avec le Path Relinking, nous observons une amélioration notable, au détriment du temps de calcul qui reste cependant acceptable si on le compare à Jump, par exemple.
Le Path Relinking, avec un nombre d'itérations adéquat, peut vraiment donner d'excellents résultats, très proches des optima connus.

Comme nous pouvons le voir sur le graphique, le comportement de l’alpha variable pour le GRASP+PR est similaire au GRASP, mais avec une meilleure exploration et un amélioration de la valeur optimale.

\vfill
\break
  % decommenter lors de la composition du EI2

% -----------------------------------------------------------------------------------------------------------------------------------------------------

% =====================================================================================
% Document : rendu du IE3
% Auteur : Xavier Gandibleux
% Année académique : 2024-2025

\section*{Livrable de l'exercice d'implémentation 3 : \\ Battle of metaheuristics}

%
% -----------------------------------------------------------------------------------------------------------------------------------------------------
%

\vspace{5mm}
\noindent
\fbox{
  \begin{minipage}{0.97 \textwidth}
    \begin{center}
      \vspace{1mm}
        \Large{Présentation succincte des choix de mise en \oe uvre de ACO à GRASP appliquée au SPP}
      \vspace{1mm}
    \end{center}
  \end{minipage}
}
\vspace{2mm}

\noindent
Pour cette étape on a choisi d'implémenter \textbf{l'Algorithme de colonies de fourmis (ACO)}, un algorithme inspirés du comportement des fourmis.

Le paradigma :
\begin{enumerate}
    \item une fourmi parcourt plus ou moins au hasard l’environnement autour de la colonie
    \item si celle-ci découvre une source de nourriture, elle rentre plus ou moins directement au nid, en laissant sur son chemin une piste de phéromones
    \item Les fourmis attirées par les phéromones auront tendance à suivre, plus ou moins directement, la piste ainsi tracée
    \item en revenant au nid, ces mêmes fourmis vont renforcer la piste
    \item si deux pistes sont possibles pour atteindre la même source de nourriture, celle étant la plus courte sera, dans le même temps, parcourue par plus de fourmis que la longue piste
    \item la piste courte sera donc de plus en plus renforcée, et donc de plus en plus attractive
    \item la longue piste, elle, finira par disparaître, les phéromones étant volatiles
    \item à terme, l’ensemble des fourmis a donc déterminé et « choisi » la piste la plus courte
\end{enumerate}

\subsubsection*{Les variables :}

\begin{itemize}
    \item \textbf{\texttt{num\_ants}} : nombre de fourmis utilisées à chaque itération pour construire des solutions candidates.

    \item \textbf{\texttt{num\_iter}} : nombre total d’itérations de l’algorithme ACO.

    \item \textbf{\texttt{alpha}} : paramètre qui contrôle l’influence des phéromones lors de la sélection des éléments par les fourmis. Plus $\alpha$ est grand, plus les phéromones guideront fortement le choix.

    \item \textbf{\texttt{beta}} : paramètre qui contrôle l’influence de la qualité heuristique (ici, le coût $C[i]$) sur le choix des éléments par les fourmis. Plus $\beta$ est grand, plus la sélection favorise les éléments de coût élevé.

    \item \textbf{\texttt{rho}} : taux d’évaporation des phéromones à chaque itération. Détermine la vitesse à laquelle l’information historique est oubliée.

    \item \textbf{\texttt{Q}} : facteur de renforcement des phéromones, proportionnel à la qualité de la solution trouvée.

\end{itemize}



\begin{algorithm}[h!]
\caption{WeightedChoice}
\label{alg:weighted_choice}
\begin{algorithmic}[1]
    \If{$A$ est vide}
        \State \textbf{erreur:} ensemble vide
    \EndIf
    \State Remplacer tout poids négatif ou non fini par $0$
    \State $W \leftarrow \sum w$
    \If{$W = 0$}
        \State \textbf{retourner} un élément choisi uniformément dans $A$
    \EndIf
    \State Calculer la somme cumulative: $cw_i = \sum_{j \le i} \frac{w_j}{W}$
    \State Tirer un nombre aléatoire $r \sim U(0,1)$
    \State Trouver le plus petit $i$ tel que $cw_i \ge r$
    \State \textbf{retourner} $A[i]$
\end{algorithmic}
\end{algorithm}

\begin{algorithm}[h!]
\caption{ConstructSolutionAnt}
\label{alg:construct_solution_ant}
\begin{algorithmic}[1]
    \State $x \leftarrow$ vecteur de $0$ (solution)
    \State $U \leftarrow$ ensemble des lignes non encore couvertes
    \State $Cnd \leftarrow$ ensemble des colonnes disponibles

    \While{$Cnd \neq \emptyset$}
        \State $F \leftarrow \emptyset$ \Comment{colonnes faisables}
        \State $W \leftarrow \emptyset$ \Comment{poids}

        \For{chaque colonne $i \in Cnd$}
            \If{$i$ ne couvre aucune ligne déjà utilisée}
                \State Ajouter $i$ à $F$
                \State $\eta_i \leftarrow \max(10^{-9}, C[i])$ \Comment{visibilité}
                \State Ajouter $(\tau[i]^\alpha \cdot \eta_i^\beta)$ à $W$
            \EndIf
        \EndFor

        \If{$F$ est vide}
            \State \textbf{break}
        \EndIf

        \State $c \leftarrow \text{WeightedChoice}(F, W)$
        \State Marquer toutes les lignes couvertes par $c$ comme utilisées
        \State $x[c] \leftarrow 1$
        \State Retirer $c$ de $Cnd$
    \EndWhile

    \State \textbf{retourner} $x$
\end{algorithmic}
\end{algorithm}

\begin{algorithm}[h!]
\caption{ACO\_SPP}
\label{alg:aco_spp}
\begin{algorithmic}[1]
    \Require Coûts $C$, matrice $A$, nombre de fourmis $m$, itérations $T$, paramètres $\alpha,\beta,\rho,Q$
    \State Initialiser les phéromones $\tau_i \leftarrow 1$ pour tout $i$
    \State $x^{best} \leftarrow$ solution nulle
    \State $z^{best} \leftarrow -\infty$

    \For{$t = 1$ \textbf{à} $T$}
        \State $S \leftarrow \emptyset$ \Comment{solutions des fourmis}
        \State $Z \leftarrow \emptyset$ \Comment{valeurs}

        \For{chaque fourmi $k = 1..m$}
            \State $x^k \leftarrow$ \textsc{ConstructSolutionAnt}$(C,A,\tau,\alpha,\beta)$
            \If{local search activée}
                \State $x^k \leftarrow$ \textsc{LocalSearch}$(x^k)$
            \EndIf
            \State $z^k \leftarrow \sum_i C[i] \cdot x^k[i]$
            \State Ajouter $x^k$ à $S$ et $z^k$ à $Z$
            \If{$z^k > z^{best}$}
                \State $x^{best} \leftarrow x^k$
                \State $z^{best} \leftarrow z^k$
            \EndIf
        \EndFor

        \State \Comment{Évaporation}
        \State $\tau_i \leftarrow (1-\rho) \cdot \tau_i$ pour tout $i$

        \State \Comment{Dépôt de phéromones des fourmis}
        \For{chaque solution $x^k$}
            \If{$z^k > 0$}
                \State $\Delta \tau \leftarrow Q \cdot z^k / \sum C$
                \For{chaque $i$ tel que $x^k[i]=1$}
                    \State $\tau_i \leftarrow \tau_i + \Delta \tau$
                \EndFor
            \EndIf
        \EndFor

        \State \Comment{Dépôt élitiste}
        \State $\Delta \tau^{elite} \leftarrow 0.5 \cdot Q \cdot z^{best} / \sum C$
        \For{chaque $i$ tel que $x^{best}[i]=1$}
            \State $\tau_i \leftarrow \tau_i + \Delta \tau^{elite}$
        \EndFor
    \EndFor

    \State \textbf{retourner} $(x^{best}, z^{best})$
\end{algorithmic}
\end{algorithm}

\subsubsection*{Déroulement de la première itération}

\subsubsection*{Initialisation}
\begin{itemize}
    \item Nombre de colonnes : $n = 9$, nombre de lignes : $m = 7$.
    \item Coûts : $C = [10, 5, 8, 6, 9, 13, 11, 4, 6]$.
    \item Matrice d'incidence $A$ : 
    \begin{itemize}
        \item $A_1 = \{1,2,3,5,7,8\}$, $A_2 = \{2,3,8\}$, $A_3 = \{2,3,8\}$,
        \item $A_4 = \{2,5,6,8,9\}$, $A_5 = \{4\}$, $A_6 = \{1,3,5,6,9\}$,
        \item $A_7 = \{2,3,7,9\}$, $A_8 = \{1,4,5,8,9\}$, $A_9 = \{?\}$ 
        % Sostituire ? con il set corretto se necessario
    \end{itemize}
    \item Phéromones initiales : $\tau_i = 1$ pour $i=1..9$.
    \item Solution courante : $\mathbf{x} = [0,0,0,0,0,0,0,0,0]$.
    \item Lignes couvertes : $U = \emptyset$.
    \item Colonnes disponibles : $Cnd = \{1,2,\dots,9\}$.
\end{itemize}

\subsubsection*{Choix de la première colonne par la fourmi}
\begin{itemize}
    \item \textbf{Colonnes faisables} : $F = \{ i \in Cnd \mid i \text{ couvre au moins une ligne non couverte} \}$.
        Supposons que toutes les colonnes sont faisables au départ : $F = \{1,2,3,4,5,6,7,8,9\}$.
    \item \textbf{Calcul des probabilités} : 
        \[
        p_i \propto (\tau_i)^\alpha \cdot (C[i])^\beta
        \]
        avec $\alpha = 1$, $\beta = 2$.  
        Exemple de calcul (approximatif) :
        \[
        p_1 \propto 1 \cdot 10^2 = 100, \quad
        p_2 \propto 1 \cdot 5^2 = 25, \dots
        \]
    \item \textbf{Sélection aléatoire selon les probabilités} : supposons que la colonne choisie est $i = 1$.
    \item \textbf{Mouvement} : ajouter la colonne $1$ à la solution.
    \item \textbf{Nouvelle solution $\mathbf{x}$} : 
    \[
    \mathbf{x} = [1, 0, 0, 0, 0, 0, 0, 0, 0]
    \] 
    (soit $\{e_1\}$).
    \item \textbf{Mise à jour des lignes couvertes} : $U \gets U \cup A_1 = \{1,2,3,5,7,8\}$.
    \item \textbf{Colonnes disponibles} : retirer la colonne 1 et toutes celles qui ne sont plus faisables à cause de lignes déjà couvertes.
\end{itemize}

\subsubsection*{Calcul de l'objectif et mise à jour des phéromones}
\begin{itemize}
    \item \textbf{Valeur de la solution} : $z = \sum_i C[i] x[i] = 10$.
    \item \textbf{Mise à jour des phéromones} :
    \begin{itemize}
        \item Évaporation : $\tau_i \gets (1-\rho) \tau_i$, ici $\rho = 0.1$, donc $\tau_i \gets 0.9 \cdot 1 = 0.9$ pour toutes les colonnes.
        \item Dépôt par la solution : pour les colonnes sélectionnées ($i=1$) : 
        \[
        \tau_1 \gets \tau_1 + \Delta\tau_1, \quad \Delta\tau_1 = \frac{Q \cdot z}{\sum C} = \frac{1 \cdot 10}{\sum C}
        \]
        \item Dépôt élitiste : idem pour la meilleure solution globale (ici même que la solution courante).
    \end{itemize}
    \item \textbf{Nouvelle valeur des phéromones} :
    \[
    \tau = [\tau_1, \tau_2, \dots, \tau_9] \approx [0.9 + \Delta\tau_1, 0.9, \dots, 0.9]
    \]
\end{itemize}

\subsubsection*{Fin de la première itération}
\begin{itemize}
    \item Solution courante : $\mathbf{x} = \{e_1\}$.
    \item Valeur : $z = 10$.
    \item Phéromones mises à jour : $\tau_1 > 0.9$, les autres $\tau_i = 0.9$.
\end{itemize}

%
% -----------------------------------------------------------------------------------------------------------------------------------------------------
%

\vspace{5mm}
\noindent
\fbox{
  \begin{minipage}{0.97 \textwidth}
    \begin{center}
      \vspace{1mm}
        \Large{Expérimentation numérique comparative GRASP vs ACO}
      \vspace{1mm}
    \end{center}
  \end{minipage}
}
\vspace{2mm}

\noindent
Paramètres de test pour les graphiques :
\begin{itemize}
    \item nbInstances = 10
    \item nbRunGrasp = 30
    \item nbIterationGrasp = 100
    \item nbDivisionRun = 10
    \item alpha = [0.0, 0.5, 1.0]
    \item beta=[0.0, 1.0, 2.0]
    \item num\_ants=30
    \item rho=0.1
    \item Q=1.0
\end{itemize}
\begin{figure}[h!]
    \centering
    \begin{minipage}{0.48\textwidth}
        \centering
        \includegraphics[width=\linewidth]{imgs/bilan_tous_runsACO.png}
        \caption{Bilan sur tous les run d'ACO}
        \label{fig:runACO}
    \end{minipage}
    \hfill
    \begin{minipage}{0.48\textwidth}
        \centering
        \includegraphics[width=\linewidth]{imgs/bilan_CPUt_tous_runsACO.png}
        \caption{CPU temps d’execution d'ACO}
        \label{fig:bilanACO}
    \end{minipage}
\end{figure}
\begin{figure}[h!]
    \centering
    \includegraphics[width=1\linewidth]{imgs/ACO.png}
    \caption{Comparison changement d'alpha et Beta sur ACO}
    \label{fig:placeholder}
\end{figure}

Les parametres : 
\begin{itemize}
    \item Alpha ($\alpha$) :
    \begin{itemize}
        \item Bas (0) : Recherche aléatoire 
        \item Haut (2+) : Stagnation rapide.
    \end{itemize}
\end{itemize}
\begin{itemize}
    \item Beta ($\beta$) : 
    \begin{itemize}
        \item Bas (0) : Convergence lente.
        \item Haut (5+) : Comportement glouton.
    \end{itemize}
\end{itemize}
\begin{itemize}
    \item Évaporation ($\rho$) : 
    \begin{itemize}
        \item Bas (0.01) : Mémoire trop longue.
        \item Haut (0.5+) : "Amnésie", l'algorithme n'apprend pas.
    \end{itemize}
\end{itemize}


\begin{longtable}{|c|c|c|}
\hline
Fichier & Solution trouvée & Temps (s) \\
\hline
pb\_1000rnd0100.dat & 67.0 & 19.77 \\
pb\_1000rnd0800.dat & 168.0 & 135.58 \\
pb\_100rnd0500.dat & 639.0 & 1.00 \\
pb\_100rnd1200.dat & 23.0 & 1.02 \\
pb\_2000rnd0400.dat & 26.0 & 389.48 \\
pb\_2000rnd0500.dat & 126.0 & 27.73 \\
pb\_200rnd0300.dat & 722.0 & 8.37 \\
pb\_200rnd1800.dat & 18.0 & 1.98 \\
pb\_500rnd0700.dat & 1127.0 & 26.58 \\
pb\_500rnd1700.dat & 182.0 & 6.64 \\
\hline
\end{longtable}



%
% -----------------------------------------------------------------------------------------------------------------------------------------------------
%

\vspace{5mm}
\noindent
\fbox{
  \begin{minipage}{0.97 \textwidth}
    \begin{center}
      \vspace{1mm}
        \Large{Discussion}
      \vspace{1mm}
    \end{center}
  \end{minipage}
}
\vspace{2mm}

\noindent
Après avoir exécuté les deux métaheuristiques sur plusieurs instances, on peut tirer les observations suivantes :

Qualité des solutions :

GRASP+PR produit de bonnes solutions dès les premières itérations, tandis qu'ACO les explore de manière plus progressive et nécessite donc un plus grand nombre d'itérations pour atteindre la solution optimale.

Variabilité et robustesse :

L'ACO présente une plus grande variabilité de solutions pour différentes exécutions, aussi en raison des paramétrages possibles.

Temps de calcul :

GRASP+PR est plus rapide sur les tailles moyennes, tandis que le temps d'exécution d'ACO augmente avec le nombre de fourmis, mais globalement, pour les meilleures solutions, ACO est l'algorithme qui prend le moins de temps.


On peut donc dire que GRASP + PR est généralement plus rapide et robuste pour obtenir des solutions de haute qualité sur des instances variées, tandis que l’ACO est plus exploratoire et paramétrable, ce qui peut permettre d’atteindre de meilleures solutions sur des instances spécifiques avec un temps de calcul suffisant.





\begingroup

    \small 
    
    \vfill 
    
    \noindent \textit{\textbf{Note :} Mon travail a été accompagné par des systèmes LLM pour les traductions, la mise en forme \LaTeX de certaines étapes/formules, et l'écriture des algorithmes. Dans mon cas, cela m'a été absolument utile comme \textbf{support didactique}, m'aidant à traduire du pseudocode en Julia ou pour l'explication de fonctions / les illustrations étape-par-étape de passages qui m'étaient peu clairs. Même lorsque les réponses n'étaient pas exactes, elles ont tout de même amélioré une recherche de solution en ligne.}
\endgroup % decommenter lors de la composition du EI3



\end{document}

