% =====================================================================================
% Document : rendu du IE2
% Auteur : Xavier Gandibleux
% Année académique : 2024-2025

\section*{Livrable de l'exercice d'implémentation 2 : \\ Métaheuristique GRASP, ReactiveGRASP et extensions}

%
% -----------------------------------------------------------------------------------------------------------------------------------------------------
%

\vspace{5mm}
\noindent
\fbox{
  \begin{minipage}{0.97 \textwidth}
    \begin{center}
      \vspace{1mm}
        \Large{Présentation succincte de GRASP appliqué sur le SPP}
      \vspace{1mm}
    \end{center}
  \end{minipage}
}
\vspace{2mm}

\noindent
Présenter l'algorithme mis en oeuvre. Illustrer sur un exemple didactique (poursuivre avec l'exemple pris en DM1). Présenter vos choix de mise en oeuvre.

 La méthode GRASP (Greedy Randomized Adaptive Search Procedure) est une métaheuristique que combine les méthodes gloutonnes et aléatoires.
 La construction d'une solution se déroule par étapes et à chacune de celles-ci, l'ensemble des morceaux de solution qu'il est possible d'ajouter est placé dans une liste appelée RCL (Restricted Candidate List). Dans la partie gloutonne cette liste est triée, mais ce n'est pas nécessairement le meilleur morceau qui est ajouté à la solution courante. 
 Pour la partie aleatoire on tire aléatoirement parmi les meilleurs possibilités le morceau à ajouter, ça permet donc de varier la forme des solutions générées mais celles-ci sont quand même de bonne qualité, puisque le choix aléatoire se fait parmi un ensemble de bons candidats. 
 La recherche locale s'applique sur la solution réalisable résultante de la phase de construction afin de voir s'il est encore possible d'améliorer cette solution.

 Je n'analyserai ci-dessous que la partie construction, car pour l'amélioration, on utilise la recherche d'échanges locaux 1-1 vue en EI1.

\begin{algorithm}[h!]
    \caption{La construction gloutonne randomisée}
    \label{alg:greedy_randomized_construction}
    \begin{algorithmic}[1]
        \State $S \leftarrow \emptyset$ \Comment{Solution courante}
        \State Initialiser l'ensemble des candidats $C$, et évaluer $u(e), \forall e \in C$
        \While {($C \neq \emptyset$)}
            \State Construire la Liste des Candidats Restreints (RCL) :
            \State \quad $u_{min} \leftarrow \min_{e \in C} u(e)$
            \State \quad $u_{max} \leftarrow \max_{e \in C} u(e)$
            \State \quad $u_{Limit} \leftarrow u_{min} + \alpha \times (u_{max} - u_{min})$ \Comment{$\alpha \in [0, 1]$ est le paramètre de gloutonnerie}
            \State \quad $RCL \leftarrow \{e \in C \mid u(e) \geq u_{Limit}\}$ \Comment{RCL contient les éléments ``suffisamment bons''}
            \State Sélectionner un élément $e$ du RCL au hasard :
            \State \quad $e \leftarrow \text{RandomSelect}(RCL)$
            \State Incorporer $e$ dans la solution :
            \State \quad $S \leftarrow S \cup \{e\}$
            \State Mettre à jour l'ensemble des candidats $C$ :
            \State \quad $C \leftarrow C \setminus \text{conflict}(\{e\})$
        \EndWhile
        \State \textbf{retourner} la solution construite $S$
    \end{algorithmic}
\end{algorithm}

\textbf{Note sur le paramètre $\alpha$} :
\begin{itemize}
    \item Si $\alpha = 0$, Il n'y a pas d'aleatoire.
    \item Si $\alpha = 1$, la sélection est totalement randomisée.
\end{itemize}

\subsection*{Exemple Didactique du GRASP}

\subsubsection*{Données de l'Instance}

\textbf{Éléments $I$ et Utilités $u(e)$ :}
$$I = \{e_1, \ldots, e_9\}$$
\begin{center}
% Correction: 10 colonnes (1 pour le titre + 9 pour e1 à e9)
\begin{tabular}{|c||c|c|c|c|c|c|c|c|c|}\hline
Élément & $e_1$ & $e_2$ & $e_3$ & $e_4$ & $e_5$ & $e_6$ & $e_7$ & $e_8$ & $e_9$ \\\hline
Utilité $u(e)$ & 10 & 5 & 8 & 6 & 9 & \textbf{13} & 11 & 4 & 6 \\\hline
\end{tabular}
\end{center}
\textbf{Contraintes de Conflit (Sous-ensembles $\mathcal{T}$)} :
\begin{itemize}
    \item $T_1 = \{e_1, e_2, e_3, e_5, e_7, e_8\}$
    \item $T_2 = \{e_2, e_3, e_8\}$
    \item $T_3 = \{e_2, e_5, e_6, e_8, e_9\}$
    \item $T_4 = \{e_4\}$
    \item $T_5 = \{e_1, e_3, e_5, e_6, e_9\}$
    \item $T_6 = \{e_2, e_3, e_7, e_9\}$
    \item $T_7 = \{e_1, e_4, e_5, e_8, e_9\}$
\end{itemize}

\subsubsection*{1. Phase de Construction Gloutonne Randomisée ($\boldsymbol{\alpha = 0.5}$)}

L'objectif est de construire une solution initiale $S$ en utilisant le critère de la Liste des Candidats Restreints (RCL).
$$ u_{Limit} = u_{min} + \alpha \times (u_{max} - u_{min}) $$

\subsubsection*{Itération 1}
\begin{itemize}
    \item $C = \{e_1, \ldots, e_9\}$. $u_{max} = 13$ ($e_6$), $u_{min} = 4$ ($e_8$).
    \item $u_{Limit} = 4 + 0.5 \times (13 - 4) = 4 + 4.5 = \mathbf{8.5}$.
    \item $\mathbf{RCL} = \{e \in C \mid u(e) \geq 8.5\} = \{e_1(10), e_5(9), e_6(13), e_7(11)\}$.
    \item \textbf{Sélection Aléatoire (Hypothèse)} : On choisit $\mathbf{e_7}$ (Utilité 11).
    \item \textbf{Mise à jour $S$} : $S = \{e_7\}$. Utilité Totale : 11.
    \item \textbf{Candidats retirés} : $e_7$ est dans $T_1$ et $T_6$. Tous les éléments de $T_1$ et $T_6$ (sauf $e_7$) sont retirés, plus $e_7$.
    $$ \text{Retiré} = \{e_7\} \cup (T_1 \setminus \{e_7\}) \cup (T_6 \setminus \{e_7\}) = \{e_7, e_1, e_2, e_3, e_5, e_8, e_9\} $$
    \item \textbf{Nouveau $C$} : $\{e_4(6), e_6(13)\}$.
\end{itemize}

\subsubsection*{Itération 2}
\begin{itemize}
    \item $C = \{e_4(6), e_6(13)\}$. $u_{max} = 13$ ($e_6$), $u_{min} = 6$ ($e_4$).
    \item $u_{Limit} = 6 + 0.5 \times (13 - 6) = 6 + 3.5 = \mathbf{9.5}$.
    \item $\mathbf{RCL} = \{e \in C \mid u(e) \geq 9.5\} = \{e_6(13)\}$.
    \item \textbf{Sélection Aléatoire} : On choisit $\mathbf{e_6}$ (Utilité 13).
    \item \textbf{Mise à jour $S$} : $S = \{e_7, e_6\}$. Utilité Totale : $11 + 13 = 24$.
    \item \textbf{Candidats retirés} : $\{e_6\}$. (Les conflits avec $e_6$ sont déjà retirés dans $C$).
    \item \textbf{Nouveau $C$} : $\{e_4(6)\}$.
\end{itemize}

\subsubsection*{Itération 3}
\begin{itemize}
    \item $C = \{e_4(6)\}$. $u_{max} = 6, u_{min} = 6$. $u_{Limit} = 6$.
    \item $\mathbf{RCL} = \{e_4(6)\}$.
    \item \textbf{Sélection Aléatoire} : On choisit $\mathbf{e_4}$ (Utilité 6).
    \item \textbf{Mise à jour $S$} : $S = \{e_7, e_6, e_4\}$. Utilité Totale : $24 + 6 = 30$.
    \item \textbf{Candidats retirés} : $\{e_4\}$.
    \item \textbf{Nouveau $C$} : $\emptyset$.
\end{itemize}

\textbf{Solution de Construction $S$} : $S = \{e_4, e_6, e_7\}$, avec une utilité de $\mathbf{30}$.

\subsubsection*{2. Phase d'Amélioration Locale (déjà analisé)}
%
% -----------------------------------------------------------------------------------------------------------------------------------------------------
%

\vspace{5mm}
\noindent
\fbox{
  \begin{minipage}{0.97 \textwidth}
    \begin{center}
      \vspace{1mm}
        \Large{Présentation succincte de Path-Relinking appliqué sur le SPP}
      \vspace{1mm}
    \end{center}
  \end{minipage}
}
\vspace{2mm}

\noindent
\begin{algorithm}
\caption{Path Relinking}
\label{alg:path_relinking}
\begin{algorithmic}[h!]
    \Ensure Meilleure Solution $\mathbf{x}_{\text{best}}$ trouvée sur le chemin, et sa valeur $z_{\text{best}}$.

    \State $\mathbf{x}_i \gets \text{copy}(\mathbf{x}_A)$ \Comment{Initialiser la solution courante}
    \State $\mathbf{x}_{\text{best}} \gets \text{copy}(\mathbf{x}_i)$
    \State $z_{\text{best}} \gets \sum(C \cdot \mathbf{x}_i)$

    \State $\text{Diff} \gets \{ i \mid \mathbf{x}_i[i] \neq \mathbf{x}_B[i] \}$ \Comment{Identifier les indices de différence}

    \While{$\text{Diff} \neq \emptyset$}
        \State Sélectionner aléatoirement un indice $i$ dans $\text{Diff}$
        \State $\mathbf{x}_i[i] \gets \mathbf{x}_B[i]$ \Comment{Appliquer le mouvement de la solution guide $\mathbf{x}_B$}
        
        \State $z_{i} \gets \sum(C \cdot \mathbf{x}_i)$
        
        \If{$z_i > z_{\text{best}}$}
            \State $\mathbf{x}_{\text{best}} \gets \text{copy}(\mathbf{x}_i)$
            \State $z_{\text{best}} \gets z_{i}$
        \EndIf
        
        \State \textbf{Optionnel :} Appliquer une Recherche Locale sur $\mathbf{x}_i$
        \State $(\mathbf{x}_{\text{LS}}, z_{\text{LS}}) \gets \text{localSearch\_1\_1}(C, A, \mathbf{x}_i)$
        
        \If{$z_{\text{LS}} > z_{\text{best}}$}
            \State $\mathbf{x}_{\text{best}} \gets \text{copy}(\mathbf{x}_{\text{LS}})$
            \State $z_{\text{best}} \gets z_{\text{LS}}$
        \EndIf

        \State $\text{Diff} \gets \{ i \mid \mathbf{x}_i[i] \neq \mathbf{x}_B[i] \}$ \Comment{Mettre à jour les indices de différence}
    \EndWhile
    
    \State Ajouter $(\mathbf{x}_{\text{best}}, z_{\text{best}})$ à $\text{EliteSet}$ \Comment{Mettre à jour l'ensemble élite}

    \State \Return $\mathbf{x}_{\text{best}}, z_{\text{best}}$
\end{algorithmic}
\end{algorithm}

\subsubsection*{Données et Solutions Élite}

L'instance SPP comporte 9 éléments ($e_1$ à $e_9$) avec les utilités suivantes :

\begin{center}
\begin{tabular}{|c||c|c|c|c|c|c|c|c|c|c|}\hline
Élément & $e_1$ & $e_2$ & $e_3$ & $e_4$ & $e_5$ & $e_6$ & $e_7$ & $e_8$ & $e_9$ & \textbf{Total} \\\hline
Utilité $u(e)$ & 10 & 5 & 8 & 6 & 9 & 13 & 11 & 4 & 6 & \\\hline
\end{tabular}
\end{center}

Nous sélectionnons les deux solutions élites suivantes (représentées par un vecteur binaire où $1$ signifie sélectionné, $0$ non sélectionné) :

\begin{itemize}
    \item \textbf{Solution de Démarrage $\mathbf{x}_A$} : $\{e_4, e_6, e_7\}$.
    $$ \mathbf{x}_A = (0, 0, 0, 1, 0, 1, 1, 0, 0) \quad \implies z_A = 6 + 13 + 11 = \mathbf{30} $$
    \item \textbf{Solution de Guidage $\mathbf{x}_B$} : $\{e_1, e_2, e_8, e_9\}$.
    $$ \mathbf{x}_B = (1, 1, 0, 0, 0, 0, 0, 1, 1) \quad \implies z_B = 10 + 5 + 4 + 6 = \mathbf{25} $$
    \item \textbf{Meilleure Solution trouvée} : $\mathbf{x}_{\text{best}} \leftarrow \mathbf{x}_A$, $z_{\text{best}} \leftarrow 30$.
\end{itemize}

\subsubsection*{Déroulement du Path Relinking}

\subsubsection*{Initialisation}
L'ensemble des indices de différence ($\text{Diff}$) est :
$$\text{Diff} = \{ i \mid \mathbf{x}_A[i] \neq \mathbf{x}_B[i] \} = \{1, 2, 4, 6, 7, 8, 9\}$$
(7 différences à résoudre pour transformer $\mathbf{x}_A$ en $\mathbf{x}_B$).

\subsubsection*{Itération 1 : Résolution de la différence $e_7$}
\begin{itemize}
    \item \textbf{Sélection Aléatoire} : $i=7$ (élément $e_7$).
    \item \textbf{Mouvement} : $e_7$ est dans $\mathbf{x}_A$ (1) mais pas dans $\mathbf{x}_B$ (0). On retire $e_7$.
    \item \textbf{Nouvelle Solution $\mathbf{x}_i$} : $(0, 0, 0, 1, 0, 1, \mathbf{0}, 0, 0)$.
    \item \textbf{Utilité $z_i$} : $30 - 11 = 19$.
    \item \textbf{Mise à Jour} : $19 \ngtr z_{\text{best}}$. Aucune amélioration.
    \item \textbf{Diff} mis à jour : $\text{Diff} = \{1, 2, 4, 6, 8, 9\}$.
\end{itemize}

\subsubsection*{Itération 2 : Résolution de la différence $e_1$}
\begin{itemize}
    \item \textbf{Sélection Aléatoire} : $i=1$ (élément $e_1$).
    \item \textbf{Mouvement} : $e_1$ n'est pas dans $\mathbf{x}_i$ (0) mais est dans $\mathbf{x}_B$ (1). On ajoute $e_1$.
    \item \textbf{Nouvelle Solution $\mathbf{x}_i$} : $(\mathbf{1}, 0, 0, 1, 0, 1, 0, 0, 0)$ (soit $\{e_1, e_4, e_6\}$).
    \item \textbf{Utilité $z_i$} : $19 + 10 = 29$.
    \item \textbf{Mise à Jour} : $29 \ngtr z_{\text{best}}$. Aucune amélioration.
    \item \textbf{Diff} mis à jour : $\text{Diff} = \{2, 4, 6, 8, 9\}$.
\end{itemize}

\subsubsection*{Itération 3 : Résolution de la différence $e_9$}
\begin{itemize}
    \item \textbf{Sélection Aléatoire} : $i=9$ (élément $e_9$).
    \item \textbf{Mouvement} : On ajoute $e_9$.
    \item \textbf{Nouvelle Solution $\mathbf{x}_i$} : $(1, 0, 0, 1, 0, 1, 0, 0, \mathbf{1})$ (soit $\{e_1, e_4, e_6, e_9\}$).
    \item \textbf{Utilité $z_i$} : $29 + 6 = \mathbf{35}$.
    \item \textbf{Mise à Jour} : $35 > z_{\text{best}}=30$.
        $$\mathbf{x}_{\text{best}} \leftarrow \{e_1, e_4, e_6, e_9\}$$
        $$z_{\text{best}} \leftarrow 35$$
    \item \textbf{Diff} mis à jour : $\text{Diff} = \{2, 4, 6, 8\}$.
\end{itemize}

\subsubsection*{Itération 4 : Résolution de la différence $e_6$}
\begin{itemize}
    \item \textbf{Sélection Aléatoire} : $i=6$ (élément $e_6$).
    \item \textbf{Mouvement} : $e_6$ est dans $\mathbf{x}_i$ (1) mais pas dans $\mathbf{x}_B$ (0). On retire $e_6$.
    \item \textbf{Nouvelle Solution $\mathbf{x}_i$} : $(1, 0, 0, 1, 0, \mathbf{0}, 0, 0, 1)$ (soit $\{e_1, e_4, e_9\}$).
    \item \textbf{Utilité $z_i$} : $35 - 13 = 22$.
    \item \textbf{Mise à Jour} : $22 \ngtr z_{\text{best}}$. Aucune amélioration.
    \item \textbf{Diff} mis à jour : $\text{Diff} = \{2, 4, 8\}$.
\end{itemize}

\subsubsection*{Itérations Finales}
Le processus se poursuit jusqu'à ce que $\mathbf{x}_i$ soit identique à $\mathbf{x}_B$.

\begin{itemize}
    \item \textbf{Itération 5} : Résoudre $e_4$ (Retrait). $\mathbf{x}_i \leftarrow \{e_1, e_9\}$. $z_i = 22 - 6 = 16$.
    \item \textbf{Itération 6} : Résoudre $e_2$ (Ajout). $\mathbf{x}_i \leftarrow \{e_1, e_2, e_9\}$. $z_i = 16 + 5 = 21$.
    \item \textbf{Itération 7} : Résoudre $e_8$ (Ajout). $\mathbf{x}_i \leftarrow \{e_1, e_2, e_8, e_9\}$. $z_i = 21 + 4 = 25$.
\end{itemize}

\subsubsection*{Conclusion du Chemin}
La solution finale du chemin est $\mathbf{x}_i = \mathbf{x}_B$ avec une utilité de 25.
%
% -----------------------------------------------------------------------------------------------------------------------------------------------------
%

\vspace{5mm}
\noindent
\fbox{
  \begin{minipage}{0.97 \textwidth}
    \begin{center}
      \vspace{1mm}
        \Large{Expérimentation numérique de GRASP}
      \vspace{1mm}
    \end{center}
  \end{minipage}
}
\vspace{2mm}

\noindent
Paramètres de test pour les graphiques :
 \begin{itemize}
     \item nbInstances       = 3
     \item nbRunGrasp        = 30
     \item nbIterationGrasp  = 100
     \item nbDivisionRun     = 10
     \item alpha = [0.0, 0.25, 0.5, 0.75, 1.0]
 \end{itemize}
 \begin{figure}[h!]
    \centering
    \begin{minipage}[b]{0.48\linewidth}
        \centering
        \includegraphics[width=\linewidth]{imgs/bilan_tous_runs_GRASP.png}
        \caption{Bilan sur tous les run}
        \label{fig:runGRASP}
    \end{minipage}
    \hfill
    \begin{minipage}[b]{0.48\linewidth}
        \centering
        \includegraphics[width=\linewidth]{imgs/bilan_CPUt_tous_runs_GRASP.png}
        \caption{CPU temps d'execution}
        \label{fig:bilanRuns}
    \end{minipage}
\end{figure}
 \begin{figure}[h!]
     \centering
     \includegraphics[width=1\linewidth]{imgs/GRASP.png}
     \caption{Comparison changement d'alpha sur GRASP}
     \label{fig:placeholder}
 \end{figure}

\vspace{40mm}
\noindent
Paramètres pour test sur best value :
\begin{itemize}
    \item alpha = 0,8
    \item itérations = 5
\end{itemize}
\vspace{2cm}
\begin{longtable}{|c|c|c|}
\hline
Fichier & Solution trouvée & Temps (s) \\
\hline
pb\_1000rnd0100.dat & 56 & 17.461596 \\
pb\_1000rnd0800.dat & 126 & 135.841186 \\
pb\_100rnd0500.dat & 627 & 0.033512 \\
pb\_100rnd1200.dat & 18 & 0.055464 \\
pb\_2000rnd0400.dat & 20 & 1094.570787 \\
pb\_2000rnd0500.dat & 119 & 68.136535 \\
pb\_200rnd0300.dat & 682 & 5.037251 \\
pb\_200rnd1800.dat & 14 & 0.377953 \\
pb\_500rnd0700.dat & 987 & 42.822087 \\
pb\_500rnd1700.dat & 164 & 6.648394 \\
\hline
\end{longtable}

Si on change alpha à 0,3 par exemple : 

\begin{longtable}{|c|c|c|}
\hline
Fichier & Solution trouvée & Temps (s) \\
\hline
pb\_1000rnd0100.dat & 40 & 32.844663 \\
pb\_1000rnd0800.dat & 130 & 135.449871 \\
pb\_100rnd0500.dat & 633 & 0.083169 \\
pb\_100rnd1200.dat & 18 & 0.058411 \\
pb\_2000rnd0400.dat & 21 & 1075.133779 \\
pb\_2000rnd0500.dat & 108 & 137.884737 \\
pb\_200rnd0300.dat & 672 & 12.449107 \\
pb\_200rnd1800.dat & 13 & 0.356583 \\
pb\_500rnd0700.dat & 1023 & 159.023518 \\
pb\_500rnd1700.dat & 145 & 9.482767 \\
\hline
\end{longtable}

Si l'on augmente le déterminisme, GRASP n'explorera pas grand-chose en dehors des résultats trouvés, au contraire, avec trop d'aléatoire, il explorera trop loin. Il est nécessaire de doser correctement alpha en fonction du type de problème.

Ex alpha sur pb\_200rnd0300 :
\begin{itemize}
    \item alpha 1 : 639 en 2 secondes
    \item alpha 0 : 678 en 24 secondes
\end{itemize}

En conclusion, comme nous pouvons le voir sur le graphique, plus nous augmentons la valeur alpha, moins il y aura d'exploration. C'est pourquoi avoir un alpha faible prend plus de temps à s'exécuter.

%
% -----------------------------------------------------------------------------------------------------------------------------------------------------
%

\vspace{5mm}
\noindent
\fbox{
  \begin{minipage}{0.97 \textwidth}
    \begin{center}
      \vspace{1mm}
        \Large{Expérimentation numérique de Path-Relinking}
      \vspace{1mm}
    \end{center}
  \end{minipage}
}
\vspace{2mm}

Paramètres de test pour les graphiques :
\begin{itemize}
    \item nbInstances = 2
    \item nbRunGrasp = 3
    \item nbIterationGrasp = 100
    \item nbDivisionRun = 10
    \item alpha = [0.0, 0.25, 0.5, 0.75, 1.0]
\end{itemize}
\begin{figure}[h!]
    \centering
    \begin{minipage}[b]{0.48\linewidth}
        \centering
        \includegraphics[width=\linewidth]{imgs/bilan_tous_runs_PR.png}
        \caption{Bilan sur tous les run GRASP+PR}
        \label{fig:runGRASP_PR}
    \end{minipage}
    \hfill
    \begin{minipage}[b]{0.48\linewidth}
        \centering
        \includegraphics[width=\linewidth]{imgs/bilan_CPUt_tous_runs_PR.png}
        \caption{CPU temps d'execution GRASP+PR}
        \label{fig:bilanGRASP_PR}
    \end{minipage}
\end{figure}
\begin{figure}[h!]
    \centering
    \includegraphics[width=1\linewidth]{imgs/GRASP_PR.png}
    \caption{Comparison changement d'alpha sur GRASP+PR}
    \label{fig:placeholder}
\end{figure}

\vspace{40cm}

\noindent
Paramètres pour les tests :
\begin{itemize}
    \item alpha = 0,8
    \item itérations = 5
\end{itemize}

\begin{longtable}{|c|c|c|c|c|}
\hline
Fichier & zmin & zmoy & zmax & Temps (s) \\
\hline
pb\_1000rnd0100.dat & 37 & 52.0 & 60 & 55.17 \\
pb\_100rnd0500.dat & 620 & 643.0 & 648 & 0.19 \\
pb\_100rnd1200.dat & 13 & 19.2 & 22 & 1.41 \\
pb\_200rnd0300.dat & 643 & 735.4 & 810 & 82.71 \\
pb\_200rnd1800.dat & 11 & 16.2 & 18 & 8.87 \\
pb\_500rnd1700.dat & 98 & 166.2 & 211 & 53.92 \\
\hline
\end{longtable}

\noindent
Le Path Relinking améliore considérablement les solutions trouvées par GRASP, mais prend beaucoup plus de temps, c'est pourquoi je n'ai pas les résultats pour tous les cas.


%
% -----------------------------------------------------------------------------------------------------------------------------------------------------
%

\vspace{5mm}
\noindent
\fbox{
  \begin{minipage}{0.97 \textwidth}
    \begin{center}
      \vspace{1mm}
        \Large{Discussion}
      \vspace{1mm}
    \end{center}
  \end{minipage}
}
\vspace{2mm}

\noindent

Le GRASP seul approche déjà une solution optimale dans un temps relativement rapide, en veillant à bien paramétrer le nombre d'itérations et $\alpha$. Avec le Path Relinking, nous observons une amélioration notable, au détriment du temps de calcul qui reste cependant acceptable si on le compare à Jump, par exemple.
Le Path Relinking, avec un nombre d'itérations adéquat, peut vraiment donner d'excellents résultats, très proches des optima connus.

Comme nous pouvons le voir sur le graphique, le comportement de l’alpha variable pour le GRASP+PR est similaire au GRASP, mais avec une meilleure exploration et un amélioration de la valeur optimale.

\vfill
\break
